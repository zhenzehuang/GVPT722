% Options for packages loaded elsewhere
\PassOptionsToPackage{unicode}{hyperref}
\PassOptionsToPackage{hyphens}{url}
\PassOptionsToPackage{dvipsnames,svgnames,x11names}{xcolor}
%
\documentclass[
  letterpaper,
  DIV=11,
  numbers=noendperiod]{scrartcl}

\usepackage{amsmath,amssymb}
\usepackage{iftex}
\ifPDFTeX
  \usepackage[T1]{fontenc}
  \usepackage[utf8]{inputenc}
  \usepackage{textcomp} % provide euro and other symbols
\else % if luatex or xetex
  \usepackage{unicode-math}
  \defaultfontfeatures{Scale=MatchLowercase}
  \defaultfontfeatures[\rmfamily]{Ligatures=TeX,Scale=1}
\fi
\usepackage{lmodern}
\ifPDFTeX\else  
    % xetex/luatex font selection
\fi
% Use upquote if available, for straight quotes in verbatim environments
\IfFileExists{upquote.sty}{\usepackage{upquote}}{}
\IfFileExists{microtype.sty}{% use microtype if available
  \usepackage[]{microtype}
  \UseMicrotypeSet[protrusion]{basicmath} % disable protrusion for tt fonts
}{}
\makeatletter
\@ifundefined{KOMAClassName}{% if non-KOMA class
  \IfFileExists{parskip.sty}{%
    \usepackage{parskip}
  }{% else
    \setlength{\parindent}{0pt}
    \setlength{\parskip}{6pt plus 2pt minus 1pt}}
}{% if KOMA class
  \KOMAoptions{parskip=half}}
\makeatother
\usepackage{xcolor}
\setlength{\emergencystretch}{3em} % prevent overfull lines
\setcounter{secnumdepth}{-\maxdimen} % remove section numbering
% Make \paragraph and \subparagraph free-standing
\ifx\paragraph\undefined\else
  \let\oldparagraph\paragraph
  \renewcommand{\paragraph}[1]{\oldparagraph{#1}\mbox{}}
\fi
\ifx\subparagraph\undefined\else
  \let\oldsubparagraph\subparagraph
  \renewcommand{\subparagraph}[1]{\oldsubparagraph{#1}\mbox{}}
\fi


\providecommand{\tightlist}{%
  \setlength{\itemsep}{0pt}\setlength{\parskip}{0pt}}\usepackage{longtable,booktabs,array}
\usepackage{calc} % for calculating minipage widths
% Correct order of tables after \paragraph or \subparagraph
\usepackage{etoolbox}
\makeatletter
\patchcmd\longtable{\par}{\if@noskipsec\mbox{}\fi\par}{}{}
\makeatother
% Allow footnotes in longtable head/foot
\IfFileExists{footnotehyper.sty}{\usepackage{footnotehyper}}{\usepackage{footnote}}
\makesavenoteenv{longtable}
\usepackage{graphicx}
\makeatletter
\def\maxwidth{\ifdim\Gin@nat@width>\linewidth\linewidth\else\Gin@nat@width\fi}
\def\maxheight{\ifdim\Gin@nat@height>\textheight\textheight\else\Gin@nat@height\fi}
\makeatother
% Scale images if necessary, so that they will not overflow the page
% margins by default, and it is still possible to overwrite the defaults
% using explicit options in \includegraphics[width, height, ...]{}
\setkeys{Gin}{width=\maxwidth,height=\maxheight,keepaspectratio}
% Set default figure placement to htbp
\makeatletter
\def\fps@figure{htbp}
\makeatother

\usepackage{booktabs}
\usepackage{longtable}
\usepackage{array}
\usepackage{multirow}
\usepackage{wrapfig}
\usepackage{float}
\usepackage{colortbl}
\usepackage{pdflscape}
\usepackage{tabu}
\usepackage{threeparttable}
\usepackage{threeparttablex}
\usepackage[normalem]{ulem}
\usepackage{makecell}
\usepackage{xcolor}
\usepackage{siunitx}

  \newcolumntype{d}{S[
    input-open-uncertainty=,
    input-close-uncertainty=,
    parse-numbers = false,
    table-align-text-pre=false,
    table-align-text-post=false
  ]}
  
\KOMAoption{captions}{tableheading}
\makeatletter
\@ifpackageloaded{tcolorbox}{}{\usepackage[skins,breakable]{tcolorbox}}
\@ifpackageloaded{fontawesome5}{}{\usepackage{fontawesome5}}
\definecolor{quarto-callout-color}{HTML}{909090}
\definecolor{quarto-callout-note-color}{HTML}{0758E5}
\definecolor{quarto-callout-important-color}{HTML}{CC1914}
\definecolor{quarto-callout-warning-color}{HTML}{EB9113}
\definecolor{quarto-callout-tip-color}{HTML}{00A047}
\definecolor{quarto-callout-caution-color}{HTML}{FC5300}
\definecolor{quarto-callout-color-frame}{HTML}{acacac}
\definecolor{quarto-callout-note-color-frame}{HTML}{4582ec}
\definecolor{quarto-callout-important-color-frame}{HTML}{d9534f}
\definecolor{quarto-callout-warning-color-frame}{HTML}{f0ad4e}
\definecolor{quarto-callout-tip-color-frame}{HTML}{02b875}
\definecolor{quarto-callout-caution-color-frame}{HTML}{fd7e14}
\makeatother
\makeatletter
\makeatother
\makeatletter
\makeatother
\makeatletter
\@ifpackageloaded{caption}{}{\usepackage{caption}}
\AtBeginDocument{%
\ifdefined\contentsname
  \renewcommand*\contentsname{Table of contents}
\else
  \newcommand\contentsname{Table of contents}
\fi
\ifdefined\listfigurename
  \renewcommand*\listfigurename{List of Figures}
\else
  \newcommand\listfigurename{List of Figures}
\fi
\ifdefined\listtablename
  \renewcommand*\listtablename{List of Tables}
\else
  \newcommand\listtablename{List of Tables}
\fi
\ifdefined\figurename
  \renewcommand*\figurename{Figure}
\else
  \newcommand\figurename{Figure}
\fi
\ifdefined\tablename
  \renewcommand*\tablename{Table}
\else
  \newcommand\tablename{Table}
\fi
}
\@ifpackageloaded{float}{}{\usepackage{float}}
\floatstyle{ruled}
\@ifundefined{c@chapter}{\newfloat{codelisting}{h}{lop}}{\newfloat{codelisting}{h}{lop}[chapter]}
\floatname{codelisting}{Listing}
\newcommand*\listoflistings{\listof{codelisting}{List of Listings}}
\makeatother
\makeatletter
\@ifpackageloaded{caption}{}{\usepackage{caption}}
\@ifpackageloaded{subcaption}{}{\usepackage{subcaption}}
\makeatother
\makeatletter
\@ifpackageloaded{tcolorbox}{}{\usepackage[skins,breakable]{tcolorbox}}
\makeatother
\makeatletter
\@ifundefined{shadecolor}{\definecolor{shadecolor}{rgb}{.97, .97, .97}}
\makeatother
\makeatletter
\makeatother
\makeatletter
\makeatother
\ifLuaTeX
  \usepackage{selnolig}  % disable illegal ligatures
\fi
\IfFileExists{bookmark.sty}{\usepackage{bookmark}}{\usepackage{hyperref}}
\IfFileExists{xurl.sty}{\usepackage{xurl}}{} % add URL line breaks if available
\urlstyle{same} % disable monospaced font for URLs
\hypersetup{
  pdftitle={Short Paper 1},
  pdfauthor={Zhenze Huang},
  colorlinks=true,
  linkcolor={blue},
  filecolor={Maroon},
  citecolor={Blue},
  urlcolor={Blue},
  pdfcreator={LaTeX via pandoc}}

\title{Short Paper 1}
\author{Zhenze Huang}
\date{}

\begin{document}
\maketitle
\ifdefined\Shaded\renewenvironment{Shaded}{\begin{tcolorbox}[breakable, borderline west={3pt}{0pt}{shadecolor}, interior hidden, boxrule=0pt, sharp corners, frame hidden, enhanced]}{\end{tcolorbox}}\fi

\hypertarget{hypothesis}{%
\subsection{Hypothesis}\label{hypothesis}}

Based on the discussions of introduction and theory, this paper proposes
three hypotheses shown as follow:

H1: On average, the closer the date of the election, the lower the voter
turnout.

\begin{tcolorbox}[enhanced jigsaw, titlerule=0mm, colframe=quarto-callout-note-color-frame, rightrule=.15mm, coltitle=black, leftrule=.75mm, breakable, colback=white, toprule=.15mm, left=2mm, bottomrule=.15mm, colbacktitle=quarto-callout-note-color!10!white, opacityback=0, arc=.35mm, bottomtitle=1mm, opacitybacktitle=0.6, title=\textcolor{quarto-callout-note-color}{\faInfo}\hspace{0.5em}{Note}, toptitle=1mm]

Rather, the further the state's closing date for voter registration is
from its election day, the lower the expected voter turnout rate will
be.

\end{tcolorbox}

H2: The more voters with a high school diploma in the state, the higher
the voter turnout in the election.

\begin{tcolorbox}[enhanced jigsaw, titlerule=0mm, colframe=quarto-callout-note-color-frame, rightrule=.15mm, coltitle=black, leftrule=.75mm, breakable, colback=white, toprule=.15mm, left=2mm, bottomrule=.15mm, colbacktitle=quarto-callout-note-color!10!white, opacityback=0, arc=.35mm, bottomtitle=1mm, opacitybacktitle=0.6, title=\textcolor{quarto-callout-note-color}{\faInfo}\hspace{0.5em}{Note}, toptitle=1mm]

This phrasing suggests that the hypothesis relates to the absolute
number of high school graduates. This is not the case. The hypothesis
relates to the \emph{proportion} of the state's population that
graduated from high school.

\end{tcolorbox}

H3: The voters in states located in the Southern states are less likely
to turnout for the election compared to voters in states located in
non-Southern states.

Specifically, I assume that there is a negative correlation between the
average deadline for voter registration and voter turnout. Specifically,
voter turnout is expected to decline as the registration deadline
approaches Election Day.

\begin{tcolorbox}[enhanced jigsaw, titlerule=0mm, colframe=quarto-callout-note-color-frame, rightrule=.15mm, coltitle=black, leftrule=.75mm, breakable, colback=white, toprule=.15mm, left=2mm, bottomrule=.15mm, colbacktitle=quarto-callout-note-color!10!white, opacityback=0, arc=.35mm, bottomtitle=1mm, opacitybacktitle=0.6, title=\textcolor{quarto-callout-note-color}{\faInfo}\hspace{0.5em}{Note}, toptitle=1mm]

Voter turnout is expected to \emph{increase} as the registration
deadline approaches election day.

\end{tcolorbox}

In contrast, a positive relationship is expected between the percentage
of high school graduates in states and voter turnout. This suggests that
higher educational attainment within a state correlates with increased
voter participation. Furthermore, it is hypothesized that, on average,
states located in the Southern region of the United States will exhibit
lower voter turnout compared to their non-Southern counterparts, while
controlling for other relevant factors. The mathematical equation shows
as follow:

\[Y = \beta_0+\beta_1(close)+\beta_2(pcthsg)+\beta_3(south)+e\]

In the equation, the dependent variable, denoted as \(Y\), signifying
the voter turnout. \(\beta_0\) stands for the intercept, while
\(\beta_1\) and \(\beta_2\) represent two independent variables,
respectively. The term \(e\) is the error term.

\hypertarget{result}{%
\subsection{Result}\label{result}}

\hypertarget{bivariate-analysis}{%
\subsubsection{\texorpdfstring{\textbf{1. Bivariate
analysis:}}{1. Bivariate analysis:}}\label{bivariate-analysis}}

Figure 1 illustrates a scatter plot depicting the relationship between
the closing date for voter registration and voter turnout. The data
presented suggest a negative correlation, indicating that as the
registration deadline approaches, voter turnout tends to decline. The
scatter plot shown as Figure 2 indicates the positive relationship
between the percentage of high school graduates and voter turnout: the
more the high school graduates, the higher the voter turnout. Figure 3
suggests that on average, voter turnout in the southern states (mean =
61.07) tends to be relatively lower than in the non-southern states
(mean = 68.18).

\begin{tcolorbox}[enhanced jigsaw, titlerule=0mm, colframe=quarto-callout-note-color-frame, rightrule=.15mm, coltitle=black, leftrule=.75mm, breakable, colback=white, toprule=.15mm, left=2mm, bottomrule=.15mm, colbacktitle=quarto-callout-note-color!10!white, opacityback=0, arc=.35mm, bottomtitle=1mm, opacitybacktitle=0.6, title=\textcolor{quarto-callout-note-color}{\faInfo}\hspace{0.5em}{Note}, toptitle=1mm]

It would also be good to see bivariate regressions for each of these
relationships.

\end{tcolorbox}

\hypertarget{multiple-regression-analysis}{%
\subsubsection{\texorpdfstring{\textbf{2. Multiple Regression
Analysis:}}{2. Multiple Regression Analysis:}}\label{multiple-regression-analysis}}

The results of the OLS model are shown in Table 1. Consistent with
Hypothesis 1, which assumed a negative relationship between the
registration deadline and voter turnout, our results indicate that, on
average, each day closer to the registration deadline is associated with
a 0.207 percentage point decrease in voter turnout, holding all other
factors constant.

\begin{tcolorbox}[enhanced jigsaw, titlerule=0mm, colframe=quarto-callout-note-color-frame, rightrule=.15mm, coltitle=black, leftrule=.75mm, breakable, colback=white, toprule=.15mm, left=2mm, bottomrule=.15mm, colbacktitle=quarto-callout-note-color!10!white, opacityback=0, arc=.35mm, bottomtitle=1mm, opacitybacktitle=0.6, title=\textcolor{quarto-callout-note-color}{\faInfo}\hspace{0.5em}{Note}, toptitle=1mm]

Error carried forward.

The \texttt{close} variable provides us with the number of days prior to
the state's election day voter registration closes. Some states allow
you to register on the day of the election (\texttt{close\ =\ 0} ).
Others require that you register at least 50 days prior to the election
day (\texttt{close\ =\ 50}). Our expectation (supported by your
analysis) is that, as the number of days prior to the election day you
need to register to vote in that election increase, the voter turnout
rate decreases.

\end{tcolorbox}

This relationship is statistically significant (p \(<\) 0.001). However,
from a substantive perspective, the observed 0.207 percentage point
decrease in turnout may not have a significant impact on the outcome of
an election, particularly in the early stages or in the competitive
multi-candidate races.

\begin{tcolorbox}[enhanced jigsaw, titlerule=0mm, colframe=quarto-callout-note-color-frame, rightrule=.15mm, coltitle=black, leftrule=.75mm, breakable, colback=white, toprule=.15mm, left=2mm, bottomrule=.15mm, colbacktitle=quarto-callout-note-color!10!white, opacityback=0, arc=.35mm, bottomtitle=1mm, opacitybacktitle=0.6, title=\textcolor{quarto-callout-note-color}{\faInfo}\hspace{0.5em}{Note}, toptitle=1mm]

Holding all else constant, I would expect more people coming out to vote
in competitive (multi-candidate) races to have a greater impact on the
result than in uncompetitive races.

\end{tcolorbox}

In line with Hypothesis 2, which predicted that a higher percentage of
high school graduates would increase voter turnout, our results confirm
this theoretical expectation. Specifically, I find that a on average,
one percentage point increase in the proportion of high school graduates
is associated with a 0.079 percentage point increase in turnout, a
relationship that is both statistically (p = 0.0252 \(<\) 0.05) and
substantively significant. This means that states with a higher
proportion of high school graduates may experience higher voter turnout.
The results suggest two key points: first, in closely contested
elections, even a marginal increase in turnout of 0.079 percentage
points could significantly affect the outcome of the election. Second,
if turnout is indeed affected by the proportion of high school
graduates, then efforts to improve education and increase high school
graduation rates could play a key role in increasing voter
participation.

Regarding Hypothesis 3, which suggests that voter turnout would be lower
in Southern states, our model results reveal both substantive and
statistically significant differences. Specifically, after controlling
for other factors, an average voter turnout in Southern states is 5.256
percentage points lower than in non-Southern states (p \textless{}
0.001). This indicates that residing in the South is associated with a
decreased likelihood of voting. Such regional disparities in voter
turnout are of considerable significance for electoral participation.

\begin{tcolorbox}[enhanced jigsaw, titlerule=0mm, colframe=quarto-callout-note-color-frame, rightrule=.15mm, coltitle=black, leftrule=.75mm, breakable, colback=white, toprule=.15mm, left=2mm, bottomrule=.15mm, colbacktitle=quarto-callout-note-color!10!white, opacityback=0, arc=.35mm, bottomtitle=1mm, opacitybacktitle=0.6, title=\textcolor{quarto-callout-note-color}{\faInfo}\hspace{0.5em}{Note}, toptitle=1mm]

Why are they of considerable significance for electoral participation?
In other words, what is the substantive significance of this finding?

\end{tcolorbox}

\hypertarget{footnote-regarding-the-robust-standard-errors-check}{%
\subsubsection{\texorpdfstring{\textbf{2.3} Footnote Regarding the
Robust Standard Errors
Check:}{2.3 Footnote Regarding the Robust Standard Errors Check:}}\label{footnote-regarding-the-robust-standard-errors-check}}

This paper employs robust standard errors to address potential
heteroscedasticity and enhance the reliability of the model. Compared to
Table 1, Table 2 shows a slight increase in standard errors, suggesting
Table 2 estimates are more reliable for regression analyses. Notably,
variables ``Closing date of registration'' (close) and Southern-state
affiliation (south) are statistically significant at the 0.05 level.
Although the percentage of high school diploma holders (pthsg) narrowly
misses this threshold, all results are interpreted with uniform
statistical significance across variables.

\hypertarget{conclusion}{%
\subsection{Conclusion}\label{conclusion}}

In summary, this study examines the empirical relationships outlined in
three hypotheses regarding voter turnout in U.S. elections. First, I
confirm a negative correlation between the proximity of registration
deadlines and turnout, consistent with Hypothesis 1. Although
statistically significant, the observed decline in turnout may not have
a substantial impact in certain electoral contexts.

\begin{tcolorbox}[enhanced jigsaw, titlerule=0mm, colframe=quarto-callout-note-color-frame, rightrule=.15mm, coltitle=black, leftrule=.75mm, breakable, colback=white, toprule=.15mm, left=2mm, bottomrule=.15mm, colbacktitle=quarto-callout-note-color!10!white, opacityback=0, arc=.35mm, bottomtitle=1mm, opacitybacktitle=0.6, title=\textcolor{quarto-callout-note-color}{\faInfo}\hspace{0.5em}{Note}, toptitle=1mm]

Why not?

\end{tcolorbox}

Second, in support of Hypothesis 2, a positive relationship was found
between the proportion of high school graduates and voter turnout. This
underscores the potential role of education in promoting civic
engagement. In addition, our results supported Hypothesis 3,
highlighting lower voter turnout in Southern states compared to
non-Southern states, indicating regional disparities in voter
participation. In addition, the study conducts a robust standard error
test to assess the reliability and address heteroscedasticity in the
model. The comparison revealed a slight increase in the standard errors
in Table 2, leading to the recommendation to use the estimates from this
table for regression analyses. Notably, ``Closing date of registration''
and Southern-state affiliation were statistically significant at the
0.05 level, while high school diploma percentage narrowly missed
significance, interpreted with uniform statistical significance across
variables.

\hypertarget{appendix}{%
\subsection{Appendix}\label{appendix}}

\includegraphics{./-Revised--Short-Paper-1_files/figure-pdf/unnamed-chunk-3-1.pdf}

\includegraphics{./-Revised--Short-Paper-1_files/figure-pdf/unnamed-chunk-4-1.pdf}

\begin{verbatim}

    Welch Two Sample t-test

data:  southern_states$vote and non_southern_states$vote
t = -9.7728, df = 130.84, p-value < 2.2e-16
alternative hypothesis: true difference in means is not equal to 0
95 percent confidence interval:
 -8.553276 -5.673440
sample estimates:
mean of x mean of y 
 61.06750  68.18086 
\end{verbatim}

\includegraphics{./-Revised--Short-Paper-1_files/figure-pdf/unnamed-chunk-5-1.pdf}

\begin{tcolorbox}[enhanced jigsaw, titlerule=0mm, colframe=quarto-callout-note-color-frame, rightrule=.15mm, coltitle=black, leftrule=.75mm, breakable, colback=white, toprule=.15mm, left=2mm, bottomrule=.15mm, colbacktitle=quarto-callout-note-color!10!white, opacityback=0, arc=.35mm, bottomtitle=1mm, opacitybacktitle=0.6, title=\textcolor{quarto-callout-note-color}{\faInfo}\hspace{0.5em}{Note}, toptitle=1mm]

Never include raw R output in your submitted work.

\end{tcolorbox}

\begin{verbatim}

Call:
lm(formula = vote ~ close + pcthsg + south, data = data)

Residuals:
     Min       1Q   Median       3Q      Max 
-18.9845  -3.6174   0.1271   3.6121  14.1879 

Coefficients:
            Estimate Std. Error t value Pr(>|t|)    
(Intercept) 66.41158    2.97056  22.357  < 2e-16 ***
close       -0.20743    0.02783  -7.452 7.13e-13 ***
pcthsg       0.07929    0.03528   2.247   0.0252 *  
south       -5.25623    0.77398  -6.791 4.73e-11 ***
---
Signif. codes:  0 '***' 0.001 '**' 0.01 '*' 0.05 '.' 0.1 ' ' 1

Residual standard error: 5.473 on 353 degrees of freedom
Multiple R-squared:  0.3261,    Adjusted R-squared:  0.3204 
F-statistic: 56.94 on 3 and 353 DF,  p-value: < 2.2e-16
\end{verbatim}

\begin{table}

\caption{Table 1. Regression Summary}
\centering
\begin{tabular}[t]{lc}
\toprule
  & (1)\\
\midrule
(Intercept) & \num{66.412}\\
 & t = \num{22.357}\\
 & SE = \num{2.971}\\
 & {}[\num{60.569}, \num{72.254}]\\
Closing date for Registration & \num{-0.207}\\
 & t = \num{-7.452}\\
 & SE = \num{0.028}\\
 & {}[\num{-0.262}, \num{-0.153}]\\
Percentage of High School Graduate & \num{0.079}\\
 & t = \num{2.247}\\
 & SE = \num{0.035}\\
 & {}[\num{0.010}, \num{0.149}]\\
South & \num{-5.256}\\
 & t = \num{-6.791}\\
 & SE = \num{0.774}\\
 & {}[\num{-6.778}, \num{-3.734}]\\
\midrule
Num.Obs. & \num{357}\\
R2 & \num{0.326}\\
R2 Adj. & \num{0.320}\\
AIC & \num{2232.8}\\
BIC & \num{2252.2}\\
Log.Lik. & \num{-1111.390}\\
F & \num{56.938}\\
RMSE & \num{5.44}\\
\bottomrule
\end{tabular}
\end{table}

\begin{verbatim}

% Table created by stargazer v.5.2.3 by Marek Hlavac, Social Policy Institute. E-mail: marek.hlavac at gmail.com
% Date and time: Mon, Apr 15, 2024 - 14:49:40
\begin{table}[!htbp] \centering 
  \caption{} 
  \label{} 
\begin{tabular}{@{\extracolsep{5pt}}lc} 
\\[-1.8ex]\hline 
\hline \\[-1.8ex] 
 & \multicolumn{1}{c}{\textit{Dependent variable:}} \\ 
\cline{2-2} 
\\[-1.8ex] & vote \\ 
\hline \\[-1.8ex] 
 close & $-$0.207$^{***}$ \\ 
  & (0.028) \\ 
  & \\ 
 pcthsg & 0.079$^{**}$ \\ 
  & (0.035) \\ 
  & \\ 
 south & $-$5.256$^{***}$ \\ 
  & (0.774) \\ 
  & \\ 
 Constant & 66.412$^{***}$ \\ 
  & (2.971) \\ 
  & \\ 
\hline \\[-1.8ex] 
Observations & 357 \\ 
R$^{2}$ & 0.326 \\ 
Adjusted R$^{2}$ & 0.320 \\ 
Residual Std. Error & 5.473 (df = 353) \\ 
F Statistic & 56.938$^{***}$ (df = 3; 353) \\ 
\hline 
\hline \\[-1.8ex] 
\textit{Note:}  & \multicolumn{1}{r}{$^{*}$p$<$0.1; $^{**}$p$<$0.05; $^{***}$p$<$0.01} \\ 
\end{tabular} 
\end{table} 
\end{verbatim}

\begin{verbatim}
            (Intercept)         close        pcthsg        south
(Intercept) 11.19992488 -0.0227674543 -0.1334932690 -1.091045050
close       -0.02276745  0.0006291880  0.0001336328 -0.003641752
pcthsg      -0.13349327  0.0001336328  0.0016401502  0.013194910
south       -1.09104505 -0.0036417517  0.0131949101  0.613357440
\end{verbatim}

\begin{table}

\caption{Table 2. Robust Standard Errors}
\centering
\begin{tabular}[t]{lc}
\toprule
  & (1)\\
\midrule
Closing date for Registration & \num{-0.21}\\
 & t = \num{-8.27}\\
 & SE = \num{0.03}\\
 & p = \vphantom{1} \num{<0.01}\\
 & {}[\num{-0.26}, \num{-0.16}]\\
Percentage of High School Graduate & \num{0.08}\\
 & t = \num{1.96}\\
 & SE = \num{0.04}\\
 & p = \num{0.05}\\
 & {}[\num{0.00}, \num{0.16}]\\
South & \num{-5.26}\\
 & t = \num{-6.71}\\
 & SE = \num{0.78}\\
 & p = \num{<0.01}\\
 & {}[\num{-6.80}, \num{-3.72}]\\
\midrule
Num.Obs. & \num{357}\\
R2 & \num{0.326}\\
R2 Adj. & \num{0.320}\\
AIC & \num{2232.8}\\
BIC & \num{2252.2}\\
Log.Lik. & \num{-1111.390}\\
F & \num{66.006}\\
RMSE & \num{5.44}\\
Std.Errors & HC3\\
\bottomrule
\multicolumn{2}{l}{\rule{0pt}{1em}+ p $<$ 0.1, * p $<$ 0.05, ** p $<$ 0.01, *** p $<$ 0.001}\\
\end{tabular}
\end{table}

\begin{verbatim}

% Table created by stargazer v.5.2.3 by Marek Hlavac, Social Policy Institute. E-mail: marek.hlavac at gmail.com
% Date and time: Mon, Apr 15, 2024 - 14:49:40
\begin{table}[!htbp] \centering 
  \caption{} 
  \label{} 
\begin{tabular}{@{\extracolsep{5pt}}lc} 
\\[-1.8ex]\hline 
\hline \\[-1.8ex] 
 & \multicolumn{1}{c}{\textit{Dependent variable:}} \\ 
\cline{2-2} 
\\[-1.8ex] & vote \\ 
\hline \\[-1.8ex] 
 close & $-$0.207$^{***}$ \\ 
  & (0.028) \\ 
  & \\ 
 pcthsg & 0.079$^{**}$ \\ 
  & (0.035) \\ 
  & \\ 
 south & $-$5.256$^{***}$ \\ 
  & (0.774) \\ 
  & \\ 
 Constant & 66.412$^{***}$ \\ 
  & (2.971) \\ 
  & \\ 
\hline \\[-1.8ex] 
Observations & 357 \\ 
R$^{2}$ & 0.326 \\ 
Adjusted R$^{2}$ & 0.320 \\ 
Residual Std. Error & 5.473 (df = 353) \\ 
F Statistic & 56.938$^{***}$ (df = 3; 353) \\ 
\hline 
\hline \\[-1.8ex] 
\textit{Note:}  & \multicolumn{1}{r}{$^{*}$p$<$0.1; $^{**}$p$<$0.05; $^{***}$p$<$0.01} \\ 
\end{tabular} 
\end{table} 
\end{verbatim}



\end{document}
